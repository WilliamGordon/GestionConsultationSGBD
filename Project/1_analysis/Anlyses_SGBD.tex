\documentclass[a4paper,11pt]{article}
\usepackage[a4paper,bindingoffset=0.2in,left=1in,right=1in,top=1in,bottom=1in,footskip=.25in]{geometry}
\usepackage{blindtext}
\usepackage{titlesec}
\usepackage[parfill]{parskip}
\usepackage{array}
\usepackage{float}
\usepackage{tabto}
\restylefloat{table}

\setcounter{secnumdepth}{4} % how many sectioning levels to assign numbers to
\setcounter{tocdepth}{4}    % how many sectioning levels to show in ToC

% https://tex.stackexchange.com/questions/78772/creating-a-bold-and-indented-title
\renewcommand\thesection{\arabic{section}} 
\renewcommand\thesubsection{\thesection.\arabic{subsection}} 
\titleformat{\section}[block]{\bfseries}{\thesection.}{1em}{} 
\titleformat{\subsection}[block]{\bfseries\hspace{2em}}{\thesubsection}{1em}{}
\titleformat{\subsubsection}[block]{\bfseries\hspace{3em}}{\thesubsubsection}{1em}{}
\titleformat{\paragraph}{\normalfont\normalsize}{\theparagraph}{1em}{}
\titlespacing*{\paragraph}{50pt}{3.25ex plus 1ex minus .2ex}{1.5ex plus .2ex}

\title{	
	\normalfont\normalsize
	\textsc{EPHEC, Bachelier en Informatique de Gestion}\\ % Your university, school and/or department name(s)
	\vspace{25pt} % Whitespace
	\rule{\linewidth}{0.5pt}\\ % Thin top horizontal rule
	\vspace{20pt} % Whitespace
	{\huge Analyse SGBD}\\ % The assignment title
	\vspace{12pt} % Whitespace
	\rule{\linewidth}{2pt}\\ % Thick bottom horizontal rule
	\vspace{12pt} % Whitespace
}

\author{William Wauters} % Your name

\date{\normalsize\today} % Today's date (\today) or a custom date

\begin{document}
	\maketitle % Print the title
	\newpage
	
	\tableofcontents
	\newpage
	
	\section{Contexte}
	\subsection{Cadre}
	Expliquer pourquoi nous faisons ce projet
	
	Ce projet est développé dans le cadre du cours de Projet de développement SGBD de Monsieur Fievez.
	
	A travers ce projet plusieurs points sont demandés à l'étudiant:
	
	\begin{itemize}
		\item Analyse d'une demande d'un client (pas toujours claire et précise)
		\item Réalisation d'une entité association
		\item Réalisation d'une base de données
		\item Réalisation de triggers et procédures stockées pour appliquer les contraintes
		\item Réalisation d'un code en 3 couches - DAL/ BL / View
		\item Utilisations de modèles et d'erreurs séparées de la couche business
		\item Accès aux données en lecture et écriture uniquement par procédures stockées
	\end{itemize}

	\subsection{Projet}
	Le projet en lui-même consiste au développement d'un système d'information permettant de faire fonctionner deux systèmes applicatifs avec des fonctionnalités différentes.
	\begin{itemize}
		\item Application permettant la gestion des maisons médicales et des médecins
		\item Application permettant la gestion des réservations des patients
	\end{itemize}
	Ces deux applications doivent utiliser une même base de données mais des utilisateurs et vues différentes.
	\newpage
	
	\section{Analyse métier}	
	\subsection{Description de la solution envisagée}
	
	Notre solution permet la gestion des réservations de consultations entre patients et médecins dans des maisons médicales. Cette solution sera composé de deux applications distinctes.
	
	Partie médecin
	
	Partie patients
	
	\subsection{Intervenants}
	
	Les intervenants prévus pour notre solutions
	\begin{itemize}
		\item Médecin: Ajoute ces disponibilités, confirme des consultations
		\item Patient: Effectue des réservations
	\end{itemize}
	
	\newpage
	
	\subsection{Fonctions attendues}
	
	\subsubsection{Exigences fonctionnelles}
	
	\paragraph{Fonctionnalité obligatoire application patient}
	
	\begin{table}[ht]
		\begin{tabular}{|m{2.5cm}|m{12.5cm}|}
			\hline
			\textbf{Code} & \textbf{Description} \\ \hline
			EF-AP-001 & Créer un nouveau patient \\ \hline
			EF-AP-002 & Sélectionner un patient dans la liste de patients \\ \hline
			EF-AP-003 & Consulter la liste des prochaines consultations \\ \hline
			EF-AP-004 & Consulter la liste des consultations confirmées \\ \hline
			EF-AP-005 & Consulter la liste des consultations non confirmées \\ \hline
			EF-AP-006 & Faire une demande de consultation \\ \hline
			EF-AP-007 & Modifier une demande de consultation \\ \hline
			EF-AP-008 & Supprimer une demande de consultation \\ \hline
		\end{tabular}
	\end{table}

	\paragraph{Fonctionnalité obligatoire application médecin}
	
	\begin{table}[ht]
		\begin{tabular}{|m{2.5cm}|m{12.5cm}|}
			\hline
			\textbf{Code} & \textbf{Description} \\ \hline
			EF-AM-001 & Créer un nouveau médecin \\ \hline
			EF-AM-002 & Sélectionner un médecin dans la liste des médecins \\ \hline
			EF-AM-003 & Visualiser son planning \\ \hline
			EF-AM-004 & Visualiser la liste de ces consultations \\ \hline
			EF-AM-005 & Visualiser la liste de ces consultations non confirmées \\ \hline
			EF-AM-006 & Confirmer une demande de consultation \\ \hline
			EF-AM-007 & Ajouter ces présences dans une maison médicale \\ \hline
			EF-AM-008 & Modifier ces présences dans une maison médicale \\ \hline
			EF-AM-009 & Visualiser la liste des maisons médicales dans lesquelles il travail \\ \hline
			EF-AM-010 & S'inscrire dans une maison médicale \\ \hline
			EF-AM-011 & Modifier durée minimum d'une consultation dans une maison médicale \\ \hline
		\end{tabular}
	\end{table}
	
	\subsubsection{Exigences non fonctionnelles}
	
	\begin{table}[ht]
		\begin{tabular}{|m{2.5cm}|m{12.5cm}|}
			\hline
			\textbf{Code} & \textbf{Description} \\ \hline
			ENF-001 & Facilité d'encodage pour l'ajout des présences du médecin \\ \hline
			ENF-002 & Être ergonomique pour l'utilisateur \\ \hline
			ENF-003 & Permettre une vue complète et simplifiée des informations \\ \hline
		\end{tabular}
	\end{table}
	
	\newpage
	
	\subsection{Contraintes business}

	Avoir une découpe en 3 couches (DAL, BL, GUI)
	
	Être divisé en deux parties
	
	Accéder à la base de données vie Entity Framework DB First
	
	Avoir une seule base de données dédiée pour les deux parties
	
	Autoriser l'accès aux tables de la base de données uniquement par l'utilisateur admin (sa)
	
	Avoir deux schéma différents (patient, médecin)
	
	Accéder aux données et les modifier via des procédures stockées spécifiques par schéma.

	\subsubsection{Point de vue technique}
	
	\subsubsection{Composition du cahier des charges}
	
	\subsubsection{Architecture du projet, à respecter}

	\newpage
	
	\section{Analyse fonctionnelle}
	
	Pas pour le création des locaux, mais bien pour les scénarios de réservations 
		
	\subsection{Diagramme USE CASES (Les principaux)}

	\subsection{Descriptions textuelles des différents USE CASES}

	\subsubsection{Application Patient}

	\paragraph{EF-AP-001 -- Faire une demande de consultation}
	
	\textbf{Titre} : Faire une demande de consultation\\
	\textbf{Résumé} : Faire une demande de consultation en fonction d'une spécialité et d'une maison médicale\\
	\textbf{Acteurs} : Patient\\
	\textbf{Date de création} : 06/04/2021\\
	\textbf{Date de mise à jours} : 06/04/2021\\
	\textbf{Version} : 1.0\\
	\textbf{Responsable} : William Wauters\\
	
	\textit{Pré-conditions} :
	\begin{itemize}
		\setlength\itemsep{-0.7em}
		\item[--] Le patient à une compte
		\item[--] Le patient est disponible
	\end{itemize}
	
	\textit{Sénario nominale} :
	
	\textit{Echainement alternatifs} :
	
	\textit{Echainement d'erreurs} :
	
	\textit{Post-conditions} :
	
	\paragraph{EF-AP-001 -- Modifier une demande de consultation}
	
	\paragraph{EF-AP-001 -- Supprimer une demande de consultation}

	\subsubsection{Application Médecin}
	
	\paragraph{EF-AM-001 -- Ajouter ces présences}

	\paragraph{EF-AM-002 -- Confirmer une consultation}
	
	\newpage
	
	\section{Contraintes fonctionnelles}
	
	\subsection{Règles d'accès et autorisation}
	
	L'application patient est uniquement accessible aux utilisateurs patients
	
	...
	
	\subsection{Règles de structure}
	
	\subsection{Règles de validation}
	
	\subsection{Règles de calcul}
	
	\newpage
	
	\section{Description des entités}
	
	\subsection{Patient}
	
	\subsection{Médecin}
	
	\subsection{Spécialité}
	
	\subsection{Maison médicale}
	
	\subsection{Consultation}
	
	\newpage
	
	\section{Schéma relationnel de la solution}
	
	\subsection{Schéma entité-association}
	
	\subsection{Schéma relationnel}
	
	\subsection{Implémentation des contraintes}
	
	\newpage
	
	\section{Analyse technique}
	
	\subsection{Technologies proposées}
	
	La base de données a été réalisée en SQL Server
	
	Les applications sont réalisées avec ASP.NET MVC
	
	Utilisation d'une API ASP.NET avec Entity Framework DB first

	\subsection{Architecture applicative}

	\subsection{Algorithmes spécifiques}

	\subsection{Erreurs}

	\newpage
	
	\section{ANNEXES}
	
\end{document}